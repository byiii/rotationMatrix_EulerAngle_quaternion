%%% Local Veriables:
%%% coding: utf-8
%%% mode: latex
%%% TeX-engine: xetex
%%% End:

\documentclass{article}
\usepackage{latexsym}
\usepackage{amsmath}
\usepackage{amssymb}

\usepackage[top=1.25in, bottom=1.25in, left=1.5in, right=1.5in]{geometry}

\usepackage{xeCJK}
\usepackage{fontspec,xltxtra,xunicode}
\setCJKmainfont[ItalicFont=KaiTi, BoldFont=SimHei]{SimSun}
\setCJKsansfont{SimHei}
\setCJKmonofont{FangSong}

\newcommand{\song}{\CJKfamily{SimSun}}
\setCJKfamilyfont{fs}{FangSong}
\newcommand{\fs}{\CJKfamily{fs}}
\setCJKfamilyfont{kai}{KaiTi}
\newcommand{\kai}{\CJKfamily{kai}}
\setCJKfamilyfont{yahei}{Microsoft YaHei}
\newcommand{\yahei}{\CJKfamily{yahei}}
\setCJKfamilyfont{hei}{SimHei}
\newcommand{\hei}{\CJKfamily{hei}}
\setCJKfamilyfont{lishu}{LiSu}
\newcommand{\lishu}{\CJKfamily{lishu}}
\setCJKfamilyfont{youyuan}{YouYuan}
\newcommand{\youyuan}{\CJKfamily{youyuan}}

\usepackage{graphicx}
\usepackage{hyperref}

\begin{document}
\author{BYIII}
\title{Quaternion, Euler Angles, and Rotation Matrix}
\date{}

\maketitle

\section{Quaternion}
The quaternion is a number system that extends the complex numbers. Quaternion was first introduced by Irish mathematician William Rowan Hamilton in 1843.

Complex numbers have one imaginary part, which is often denoted as $i$. Quaternions extend to three imaginary parts, $i$, $j$, and $k$. Formally, quaternions is a four-dimensional vector space over the real number. The quaternions set \textbf{H} has three operations: addition, scalar multiplication, and quaternion multiplication.

\subsection*{Multiplication of Basic Elements}
Firstly, and most importantly, the multiplication of basic elements of quaternions:
\begin{displaymath}
i^2 = j^2 = k^2 = ijk = -1
\end{displaymath}
where $i$, $j$, $k$ denote the basis elements of the imaginary parts of \textbf{H}. And from the above definition, we can derive:
\begin{displaymath}
\begin{split}
ij = k, &\quad ji = -k, \\
jk = i, &\quad kj = -i, \\
ki = j, &\quad ik = -j.
\end{split}
\end{displaymath}
Then, we should notice that the multiplication between quaternions is not commutative. And the multiplication between two of imaginary bases $i$, $j$, $k$ following the \textit{i->j->k->i->...} order loop will result in a positive of the missing one, if not, then the negative of the missing imaginary basis.

\subsection*{Addition}
The Addition between two quaternions $\mathbf{q}_1 = a_1 + ib_1+jc_1+kd_1$ and $\mathbf{q}_2 = a_2 + ib_2+jc_2+kd_2$ is just the same as the linear vector space:
\begin{displaymath}
\mathbf{q}_1 + \mathbf{q}_2 = (a_1+a_2) + i(b_1+b_2) + j(c_1+c_2) + k(d_1+d_2);
\end{displaymath}

\section{Quaternion and Rotation Matrix}

\section{Euler Angles}

\section{Euler Angles and Rotation Matrix}

\section{Euler Angles and Quaternion}

\end{document}
